\documentclass[a4paper,12pt]{article}
\usepackage[english,vietnamese]{babel}
\usepackage{amsmath}
\usepackage{booktabs}
\usepackage{lmodern}
\usepackage{hyperref}
\usepackage{lmodern}
\usepackage[nottoc,numbib]{tocbibind}
\renewcommand{\thefootnote}{\fnsymbol{footnote}}

\begin{document}
\setcounter{page}{0}
\thispagestyle{empty}
\vspace*{\stretch{1}}
\begin{flushright}
  \setlength{\baselineskip}{1.4\baselineskip}
\textbf{\Huge Color Image Processing}
  \noindent\rule{\textwidth}{5pt}
  \emph{\Large Digital Image Processing}
  \vspace{\stretch{1}}

  \textbf{by Trần Minh Hiếu, Nguyễn Gia Phong, Nguyễn Văn Tùng,\\
          Nguyễn An Thiết and Nguyễn Thành Vinh\\}
  \selectlanguage{english}
  \today
\end{flushright}
\vspace*{\stretch{2}}
\pagebreak

\selectlanguage{english}
\tableofcontents
\pagebreak

\section{Introduction}
\subsection{Brief Description}
Color images existed long before the rise of computing and digital image
processing.  While most techniques of monochrome image processing such as
blur and edge detection can be directly applied to color images, others
require modification.  Furthermore, there exists procedures specific only
to color images.  In this project, we try to implement some of these
techniques and note down our findings.

This report is licensed under a CC BY-SA 4.0 license, while the source code
is available on GitHub\footnote{\url{https://github.com/McSinyx/recipe}}
under AGPLv3+.

\selectlanguage{vietnamese}
\subsection{Authors and Credits}
The work has been undertaken by group number 8, whose members are listed
in the following table.
\begin{center}
  \begin{tabular}{c c}
    \toprule
    Full name & Student ID\\
    \midrule
    Trần Minh Hiếu & BI9-101\\
    Nguyễn Gia Phong & BI9-184\\
    Nguyễn Văn Tùng & BI9-229\\
    Nguyễn An Thiết & BI8-174\\
    Nguyễn Thành Vinh & BI9-187\\
    \bottomrule
  \end{tabular}
\end{center}

We would like to express our special thanks to Dr.~Nghiêm Thị Phương,
whose lectures gave us basic understanding on the key principles of
digital image processing.  The color image processing lecture notes from
the UMSL's CS 5420 course~\cite{cs5420} also help us gain initial intuition
on the matter.

\newpage
\selectlanguage{english}
\section{Color Spaces}

\section{Color Image Enhancements}

\section{Pseudo Color Rendering}

\section{Conclusion}

\begin{thebibliography}{69}
  \bibitem{cs5420} Sanjiv~K.~Bhatia.
    \href{http://www.cs.umsl.edu/~sanjiv/classes/cs5420/lectures/color.pdf}
         {``Color Image Processing''}.
    \emph{Cmp Sci 5420: Digital Image Processing}.
    University Of Missouri---St.~Louis, Fall 2018.
\end{thebibliography}
\end{document}
